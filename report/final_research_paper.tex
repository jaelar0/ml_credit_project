\documentclass[12pt,a4paper]{article}

% Packages for enhanced functionality
\usepackage[utf8]{inputenc} % Encoding
\usepackage{amsmath,amssymb} % Math packages
\usepackage{graphicx} % For including graphics
\usepackage{hyperref} % For hyperlinks
\usepackage{setspace} % For spacing
\usepackage{geometry} % Page layout
\geometry{margin=1in}

% Title, Author, and Date
\title{Machine Learning Methods for Classifying Bad Borrowers in a Credit Portfolio \\[0.5cm]} 
\author{Jorge Reyes \\[0.25cm] \text{University of Tennessee, Knoxville} \\[0.25cm] \text{COSC 522 Machine Learning} \\[0.25cm]} 
\date{December 3, 2024}

\begin{document}

% Title Page
\maketitle
\clearpage % Starts a new page after the title page

% Table of Contents
\tableofcontents
\newpage

% Section 1: Introduction
\section{Introduction}

% Abstract
\begin{abstract}
    Financial Services companies  chose to do a credit card dataset with the goal of 
    creating a model that will predict whether customers are 
    'good' or 'bad' borrowers. The dataset contains enough features 
    to build a classification model on the data. We will start by 
    identifying what a 'bad' borrower is by looking at the 
    portion of the dataset indicating whether or not a customer has missed 
    payments throughout the lifetime of the loan. The dataset contains two files, 
    contains information 
    monthly status of a loan by customer for the entirety of the loan. 
    contains fields such as employment status, annual income 
    and other pieces of data we can use as our features to classify someone as 
    'good' or 'bad'. Since this is a classification problem, we will be using a 
    supervised learning approach. We plan to use models such as Logistic Regression, 
    Random Forest, and Gradient Boosting. The benefit to completing this project is that 
    it would provide a bank or other financial institution with a credit scoring model 
    that can assist in extending credit to customers and assessing risk of their loan 
    portfolio.
\end{abstract}
\newpage

% Section 2: Literature Review
\section{Literature Review}
Summarize existing research related to your topic. Highlight the gaps in the literature that your research aims to fill.

% Section 3: Methodology
\section{Methodology}
Describe the methods used in your research. Include details about data collection, experimental design, tools, or techniques employed.

% Section 4: Results
\section{Results}
Present the findings of your research. Use tables, figures, and charts to illustrate your results if necessary.

% Section 5: Discussion
\section{Discussion}
Interpret your results and discuss their implications. Compare your findings with those of previous studies.

% Section 6: Conclusion
\section{Conclusion}
Summarize the key points of your research. State the main conclusions and suggest potential areas for future research.

% References
\section*{References}
\bibliographystyle{plain}
\bibliography{references}

% Note: Create a 'references.bib' file for your bibliography entries.

\end{document}
